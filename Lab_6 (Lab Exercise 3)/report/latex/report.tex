% Atom auto build latex -> Ctrl+Alt+B
% Atom check spelling Ctrl+:
% To use latex python scripts: cd ~/Dropbox/GeekFiles/Latex/ && source ~/Coding/venv/bin/activate

\documentclass{article}
\usepackage[utf8]{inputenc}
\usepackage{float}
\usepackage[pdftex]{graphicx}
\usepackage{url}
\usepackage{listings}               % Code type bash
\usepackage{subfig}                 % Images in a row

\title{VIS - Lab Exercise 3 \\ \bigskip Analysis of an infographic}
\author{\\ Carlos García}
\date{May 2017}

\begin{document}
\maketitle \newpage \tableofcontents \newpage
% ------------------------------------------------------------------------------
\section{Project description}
The goal of this activity is to study an infographic and analyze its design and characteristics.
You need to deliver a written document with the analysis that has to include the following 4
points:
\begin{enumerate}
  \item A general description of the infographic, explaining its context and where it has been taken from.
  \item An analysis of all the resources used in the infographic (line bars, pie charts, ...), what data are they representing, and what message do they want to communicate.
  \item A detailed analysis of the application of Tufte’s principles in the infographic. In the cases where there is a disagreement between a principle and the infographic, explain with detail what part of the infographic is incorrect, and how it should be fixed.
  \item A list of improvements that you think can be made to increase the quality of the visualization.
\end{enumerate}
% ------------------------------------------------------------------------------
\section{}
% ------------------------------------------------------------------------------
\section{}
% ------------------------------------------------------------------------------
\section{}
% ------------------------------------------------------------------------------
\begin{figure}[H]
  \centering
  \includegraphics[width=1\textwidth]{"/home/cj/Downloads/sample-image"} %not use extension
  \caption{Caption \cite{alias}}
  % \label{Fig:if_necessary}
\end{figure}

% ------------------------------------------------------------------------------
\newpage
\bibliographystyle{unsrt}   %unsrt by appearance
% Create also a file in this folder called "references.bib" and can paste the latex info as Wikipedia use them in "Cite this page" section.
\bibliography{references}
% ------------------------------------------------------------------------------
\end{thebibliography}
\end{document}
